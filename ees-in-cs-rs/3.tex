\documentclass[12pt, letterpaper]{article}
\usepackage[utf8]{inputenc}
\usepackage{geometry}
\usepackage{fancyhdr}
\usepackage{hyperref}
\usepackage{mathptmx}

\geometry { left=25mm, right=25mm, bindingoffset=0mm, top=20mm, bottom=20mm }
\pagestyle{fancy}
\fancyhf{}
\setlength{\headheight}{15pt}
\lhead{Clavianus Juneardo – 2001539682}
\rhead{EES IN CS RS – 3}

\hypersetup{colorlinks=true, linkcolor=blue, urlcolor=blue}
\urlstyle{rm}

\begin{document}

\section*{Questions}
\begin{enumerate}
    \item Apa kendala yang anda hadapi terkait dengan pengerjaan proyek / pekerjaan / bisnis dalam hal soft skills yang sudah anda jelaskan sebelumnya?
    \item Apa upaya anda dalam memperdalam / meningkatkan soft skills tersebut untuk menyelesaikan masalah / mengerjakan proyek / pekerjaan / bisnis anda?
\end{enumerate}

\section*{Answers}
\begin{enumerate}
    \item kendala yang saya hadapi terkait dengan pekerjaan dalam hal soft skills
        \begin{enumerate}
            \item Self management: sulit menentukan tenggang waktu pekerjaan yang diberikan
            \item Initiative \& enterprise: sulit mengimplementasi sesuatu yang baru dikarenakan tenggang waktu yang ada
            \item Problem solving \& decision making: sulit melihat corner-case yang ada jika berhadapan dengan masalah yang baru saya alami
        \end{enumerate}
    \item Upaya saya dalam meningkatkan soft skills untuk penyelesaian masalah adalah lebih memperhatikan komunikasi yang mendetail.
    Misalnya jika saya sulit dalam menentukan tenggang waktu pada suatu pekerjaan, saya harus mengkonsultasikan kepada PM sehingga, PM dapat membantu saya memberikan pendapatnya.
    Komunikasi juga sangat penting dalam pembahasan suatu masalah untuk bertukar pikiran.
\end{enumerate}

\end{document}
