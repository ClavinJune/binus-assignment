\documentclass[12pt, letterpaper]{article}
\usepackage[utf8]{inputenc}
\usepackage{geometry}
\usepackage{fancyhdr}
\usepackage{hyperref}
\usepackage{mathptmx}

\geometry { left=25mm, right=25mm, bindingoffset=0mm, top=20mm, bottom=20mm }
\pagestyle{fancy}
\fancyhf{}
\setlength{\headheight}{15pt}
\lhead{Clavianus Juneardo – 2001539682}
\rhead{EES IN CS RS – 6}

\hypersetup{colorlinks=true, linkcolor=blue, urlcolor=blue}
\urlstyle{rm}

\begin{document}

\section*{Questions}
\begin{enumerate}
    \item Jelaskan kegiatan anda terkait dengan peningkatan soft skills yang anda miliki!
    \item Jelaskan peranan soft skills dalam mendukung keberhasilan proyek anda!
    
    *dijawab pada bulan terakhir enrichment program
    \item Jelaskan peningkatan soft skills yang anda alami!
    
    *dijawab pada bulan terakhir enrichment program
\end{enumerate}

\section*{Answers}
\begin{enumerate}
    \item Soft skill saya terkait Problem Solving dan Decision makin cukup meningkat setelah melakukan pengembangan sistem autentikasi, karena desain sistem autentikasi ini sepenuhnya diberikan kepada saya. Tentu saja saya tidak sendiri dalam mengerjakan projek ini. Pengembang projek ini ada 2 orang termasuk saya dimana kita harus berdiskusi untuk menemukan desisi yang tepat yang harus diambil agar optimalnya sistem yang sedang dibuat ini.
    \item Soft skill sangat berperan terhadap keberhasilan sebuah proyek. Pengambilan keputusan yang tepat untuk melakukan pengembangan sistem sangat menentukan jalannya dan hasil dari sebuah sistem yang sedang dikembangkan. Mulai dari perencanaan, hingga penyelesaian sebuah program sangat bergantung terhadap kerja sama tim dan komunikasi didalamnya.
    \item Peningkatan soft skill yang saya alami meliputi pengambilan keputusan yang tepat, komunikasi dan self management. Di masa - masa pandemi seperti ini, saya lebih sering bekerja di rumah dimana komunikasi sangat diuji dalam pengembangan sistem.
\end{enumerate}

\end{document}
