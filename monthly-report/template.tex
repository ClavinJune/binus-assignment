\documentclass{monthlyReport}

\setMonth{August}

\begin{document}
\section{Pendahuluan}
    \subsection{Profil Perusahaan}
        \subsubsection{Informasi Umum}
            \p {
                OKHOME (PT OKHOME Mobile Indonesia) merupakan layanan pembersih rumah di tanah air yang didirikan oleh founder asal Korea Selatan. Saat ini OKHOME baru melayani wilayah Jakarta dan sekitarnya, OKHOME melayani baik pasar B2B dan B2C. Staff pembersih rumah yang disebut consultant juga melewati pelatihan - pelatihan yang ada sehingga dapat memenuhi standard kebersihan yang telah ditetapkan.
            }

        \subsubsection{Sejarah}
            \p {
                Mengawali perjalanan pada tahun 2016 dengan nama OKHOME sebagai layanan pembersih rumah di Indonesia. Diawali dari layanan pembersihan rumah, kedepannya OKHOME akan membentangkan bisnisnya terkait dengan life-style hospitality.
            }
            
    \subsection{Posisi dan Peran Mahasiswa}
        \p {
            Saya diposisikan sebagai software engineer intern.
            Tanggung jawab saya adalah mengembangkan dan memelihara cloud infrastructure dan infrastructure internal tools yang digunakan oleh tim. Di dalam tim ini saya bekerja sebagai devops dan membantu dibagian backend.
        }

\break

\section{Laporan Kegiatan}
\subsection{Proses Kegiatan Bisnis}
    \p {
        OKHOME menyediakan platform untuk pemesanan layanan pembersihan rumah baik di pasar B2B maupun B2C di website OKHOME yang berdomain di \\*\url{https://okhome.id} maupun di aplikasi android dan iOS yang dapat diinstall melalui playstore dan appstore. Staff pembersih rumah yang disebut consultant ini juga sudah dilatih sehingga dapat memenuhi standard kebersihan yang ditetapkan oleh OKHOME.
    }
    
\subsection{Kegiatan di Perusahaan dan Pencapaian Learning Objectives}
    \subsubsection{Teknikal Kompetensi}
        \begin{modlist}
            \item Dapat melakukan provisioning dan pemeliharan cloud infrastructure
            \item Dapat melakukan pemeliharaan reliabilitas dan sekuritas situs
            \item Dapat enggunakan dan melakukan konfigurasi terhadap CNCF tools
            \item Dapat mengoptimisasi pengeluaran dalam penggunaan cloud provider
        \end{modlist}
        
    \subsubsection{Softskill Kompetensi}
        \begin{modlist}
            \item Self development
            \item Teamwork
            \item Problem solving and decision making
            \item Understanding the user requirement
        \end{modlist}
        
    \subsubsection{Project yang dikerjakan}
        \p {
            Saya mengerjakan projek - projek berkatian dengan infrastructure mulai dari melakukan provisioning dan maintenance pada AWS cloud, instalasi CI/CD tools, infrastructure monitoring application, application monitoring and logging, API Gateway, Reversed Proxy, VPN, dan container orchestration. Saya juga mengerjakan projek sistem autentikasi dan membuat pustaka internal dengan bahasa go.
        }

\subsection{Penuntasan Tugas dan Penanganan Masalah}
    \getFromFile{August/timeline}

\section{Penutup}
    \subsection{Kesimpulan}
        \getFromFile{August/kesimpulan}

\end{document}