\documentclass[12pt, letterpaper]{article}
\usepackage[utf8]{inputenc}
\usepackage{geometry}
\usepackage{fancyhdr}
\usepackage{hyperref}
\usepackage{mathptmx}

\geometry { left=25mm, right=25mm, bindingoffset=0mm, top=20mm, bottom=20mm }
\pagestyle{fancy}
\fancyhf{}
\setlength{\headheight}{15pt}
\lhead{Clavianus Juneardo – 2001539682}
\rhead{IN CS RS EN – 6}

\hypersetup{colorlinks=true, linkcolor=blue, urlcolor=blue}
\urlstyle{rm}

\begin{document}

\section*{Questions}
\begin{enumerate}
    \item Jelaskan apa saja kegiatan yang anda lakukan di bulan ini yang mendorong progres pekerjaan/proyek/bisnis anda!
    \item Jelaskan upaya anda yang terkait dengan peningkatan kompetensi teknis yang anda miliki!
    
    (Kompetensi teknis adalah kemampuan teknis sesuai dengan bidang ilmu jurusan anda)
    \item Jelaskan perkembangan hasil pekerjaan/proyek/bisnis yang anda peroleh di bulan ini!
    \item Berikan rangkuman proyek yang telah anda selesaikan dan hubungannya dengan kompetensi teknis yang anda miliki dan teori pendukung yang anda ketahui!
    
    (Kompetensi teknis adalah kemampuan teknis sesuai dengan bidang ilmu jurusan anda)

    *dijawab pada bulan terakhir enrichment program
    \item Jelaskan peningkatan kompetensi teknis yang anda alami!
    
    (Kompetensi teknis adalah kemampuan teknis sesuai dengan bidang ilmu jurusan anda)
    
    *dijawab pada bulan terakhir enrichment program
\end{enumerate}

\section*{Answers}
\begin{enumerate}
    \item Kegiatan yang saya lakukan di bulan ini sebagian besar terkait dengan pengembangan sistem autentikasi dan pustaka internal dengan bahasa go. Mulai dari desain sistem hingga pembuatan skrip untuk migrasi dan deployment menggunakan Go, Ansible, Bash dan Postgresql.
    \item Upaya yang saya lakukan terkait dengan peningkatan kompetensi teknis salah satunya adalah membaca artikel - artikel terkait technical development seperti \url{https://dev.to} juga membaca dokumentasi terkait software yang sedang digunakan.
    \item Perkembangan hasil pekerjaan yang diperoleh di bulan ini merupakan tersajinya sistem autentikasi di lingkungan staging (sandbox) dan siapnya SDK yang diperlukan di bahasa Java.
    \item proyek yang sudah saya selesaikan dibulan ini adalah instalasi Nexus Repository Manager, pembuatan pustaka lokal dengan menggunakan bahasa Go, sistem autentikasi, dan SDK untuk menggunakan sistem autentikasi menggunakan bahasa Java. Semua proyek tersebut membantu saya mengembangkan diri lagi dan mendalami pengembangan sistem perangkat lunak yang berhubungan dengan bidang ilmu jurusan saya saat ini yaitu Computer Science.
    \item kemampuan teknis saya cukup meningkat mulai dari pengembangan sistem menggunakan bahasa Go secara profesional, karena sebelumnya saya menggunakan bahasa Go ini hanya untuk pengembangan diri saja.
\end{enumerate}

\end{document}
