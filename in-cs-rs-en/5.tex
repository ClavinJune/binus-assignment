\documentclass[12pt, letterpaper]{article}
\usepackage[utf8]{inputenc}
\usepackage{geometry}
\usepackage{fancyhdr}
\usepackage{hyperref}
\usepackage{mathptmx}

\geometry { left=25mm, right=25mm, bindingoffset=0mm, top=20mm, bottom=20mm }
\pagestyle{fancy}
\fancyhf{}
\setlength{\headheight}{15pt}
\lhead{Clavianus Juneardo – 2001539682}
\rhead{IN CS RS EN – 5}

\hypersetup{colorlinks=true, linkcolor=blue, urlcolor=blue}
\urlstyle{rm}

\begin{document}

\section*{Questions}
\begin{enumerate}
    \item Jelaskan apa saja kegiatan yang anda lakukan di bulan ini yang mendorong progres pekerjaan/proyek/bisnis anda!
    \item Jelaskan upaya anda yang terkait dengan peningkatan kompetensi teknis yang anda miliki!
    
    (Kompetensi teknis adalah kemampuan teknis sesuai dengan bidang ilmu jurusan anda)
    \item Jelaskan perkembangan hasil pekerjaan/proyek/bisnis yang anda peroleh di bulan ini!
    \item Berikan rangkuman pekerjaan/proyek/bisnis yang telah anda selesaikan dan hubungannya dengan kompetensi teknis yang anda miliki dan teori pendukung yang anda ketahui!
    
    (Kompetensi teknis adalah kemampuan teknis sesuai dengan bidang ilmu jurusan anda)

    *dijawab pada bulan terakhir enrichment program
    \item Jelaskan peningkatan kompetensi teknis yang anda alami!
    
    (Kompetensi teknis adalah kemampuan teknis sesuai dengan bidang ilmu jurusan anda)
    
    *dijawab pada bulan terakhir enrichment program
\end{enumerate}

\section*{Answers}
\begin{enumerate}
    \item Kegiatan yang saya lakukan di bulan ini sebagian besar terkait dengan enhancement cloud infrastructure yang ada menjadi lebih optimized dari segi harga maupun fungsionalitas juga security. Mulai dari instalasi OpenVPN dan membatasi akses kedalam aplikasi dari jaringan luar kecuali menggunakan VPN, melakukan instalasi VPN Client ke router MikroTik, dan instalasi Envoyproxy sebagai Reverse Proxy.
    \item Upaya yang saya lakukan terkait dengan peningkatan kompetensi teknis salah satunya adalah membaca artikel - artikel terkait technical development seperti \url{https://dev.to} juga membaca dokumentasi terkait software yang sedang digunakan.
    \item Perkembangan hasil pekerjaan yang diperoleh di bulan ini merupakan terinstalnya OpenVPN secara baik dan bisa digunakan di sistem operasi yang umum seperti Linux, MacOS, dan Windows. terimplementasinya service discovery, load balancer dan metode failover dengan menggunakan Envoyproxy dan membatasi akses kedalam aplikasi dari jaringan luar.
    \item *dijawab pada bulan terakhir enrichment program
    \item *dijawab pada bulan terakhir enrichment program
\end{enumerate}

\end{document}
