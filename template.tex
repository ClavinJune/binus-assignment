\documentclass{monthlyReport}

\setMonth{March}

\begin{document}
\section{Pendahuluan}
    \subsection{Profil Perusahaan}
        \subsubsection{Informasi Umum}
            \p {
                AiryRooms (PT. Airy Nest Indonesia) merupakan Accomodation Network Orchestrator (ANO) berbasis teknologi yang menawarkan kamar hotel dan tiket pesawat dengan harga terjangkau.
                Airyrooms bermitra dengan berbagai hotel budget terbaik di seluruh Indonesia dengan menawarkan pengelolaan dan pengoperasian hotel dengan berbagai macam tipe.
                Dibawah brand AiryRooms tingkat kenyamanan dan kebersihan hotel-hotel tersebut akan distandarisasi sehingga dapat meningkatkan okupansi dan pemasukan.
                PT. Airy Nest Indonesia telah bekerja sama dengan lebih dari 1000 hotel yang tersebar di berbagai kota di Indonesia.
            }

        \subsubsection{Sejarah}
            \p {
                Mengawali perjalanan pada tahun 2015 dengan nama AiryRooms sebagai jaringan hotel budget, AiryRooms telah berkembang menjadi solusi managemen hotel terbesar di Indonesia.
                AiryRooms memiliki konsep Virtual Hotel Operator dimana AiryRooms mengelola dan melakukan standarisasi hotel-hotel budget tersebut untuk selanjutnya di pasarkan dengan brand AiryRooms.
            }
            
    \subsection{Posisi dan Peran Mahasiswa}
        \p {
            Saya diposisikan sebagai software engineer di tim core dengan domain reservasi.
            Tanggung jawab saya didalam domain ini adalah mengembangkan dan memelihara internal tools yang digunakan oleh tim dan customer service.
            Didalam tim core saya bekerja sebagai back-end dan front-end developer.
        }

\break

\section{Laporan Kegiatan}
\subsection{Proses Kegiatan Bisnis}
    \p {
        Airyrooms menyediakan platform untuk pemesanan kamar hotel dan pemesanan tiket penerbangan domestik di website AiryRooms yang berdomain di
        \\*\url{https://airyrooms.com} maupun di aplikasi android dan iOS yang dapat di install melali playstore dan appstore.
        Kamar hotel yang dijual di AiryRooms merupakan hotel yang sudah bekerja sama dengan AiryRooms.
        Selain melakukan penjualan kamar di platform AiryRooms sendiri, AiryRooms juga bekerja sama dengan Online Travel Agent lain seperti Traveloka, PegiPegi, Expedia dan Booking.com untuk melakukan penjualan kamar.
    }
    
\subsection{Kegiatan di Perusahaan dan Pencapaian Learning Objectives}
    \subsubsection{Teknikal Kompetensi}
        \begin{modlist}
            \item Merubah kebutuhan user kedalam bentuk kode java
            \item Merubah design user interface kedalam bentuk kode ReactJS
            \item Menggunakan dan mengoperasikan database PostgreSQL
            \item Merubah design interface email reservasi kedalam bentuk PDF menggunakan Thymeleaf
        \end{modlist}
        
    \subsubsection{Softskill Kompetensi}
        \begin{modlist}
            \item Self development
            \item Teamwork
            \item Problem solving and decision making
            \item Developing/coding solution based on application design
            \item Ability to validate the application before deployment 
            \item Understanding the user requirement
        \end{modlist}
        
    \subsubsection{Project yang dikerjakan}
        \p {
            Saya mengerjakan Centralized Reservation System sebagai tempat untuk penyimpanan reservasi kamar hotel yang terpusat dan mengubah design email reservasi dari design menjadi html untuk dikirimkan lewat email.
        }

\subsection{Penuntasan Tugas dan Penanganan Masalah}
    \getFromFile{March/timeline}

\section{Penutup}
    \subsection{Kesimpulan}
        \getFromFile{March/kesimpulan}

\end{document}